\question{
  \item Pretende-se comparar duas abordagens para providenciar multicast na camada de
  aplicação: (1) emulação com encaminhamento unicast; (2) encaminhamento multicast.
  Considere uma fonte e $32$ destinos. A fonte está interligada com os destinos através
  de uma árvore binária de encaminhadores, tendo a fonte com raiz. O custo de uma
  abordagem multicast é o número de datagramas que têm que ser enviados nessa abordagem
  por forma a alcançar todos os destinos.
}

\begin{enumerate}[leftmargin=\labelsep]
  \subquestion{
  \item Determine os custos das duas abordagens para multicast.
        }

        Considerando uma abordagem via árvore binária, com 32 destinos possíveis,
        vamos ter, claro, $\log_2{32} = 5$ níveis de encaminhadores; neste sentido:

        \begin{itemize}
          \item Para a primeira abordagem, emulação com encaminhamento unicast,
                o custo corresponderá ao produto entre o número de níveis na árvore e
                o número de destinos possíveis, já que cada mensagem irá seguir um caminho
                específico por destino; existirá um datagrama por cada
                vez que um encaminhador \textit{encaminha} um datagrama para um destino;
          \item Para a segunda abordagem, encaminhamento multicast, o custo corresponderá
                ao número de arestas na árvore, já que cada encaminhador irá fazer
                \textit{broadcast} de um datagrama para todos os seus filhos.
        \end{itemize}

        Assim sendo, podemos dizer:

        \begin{itemize}
          \item Primeira abordagem: $D \times \log_2{D} = 32 \times 5 = 160$;
          \item Segunda abordagem: $\sum_{i=1}^{\log_2{D}} 2^i = 62$.
        \end{itemize}

        \subquestion{
  \item Encontre a topologia de rede, incluindo a fonte, os destinos, e tantos encaminhadores
        quanto quiser, que maximiza a razão entre o custo da abordagem por emulação com
        encaminhamento unicast e o custo da abordagem por encaminhamento multicast na camada.
        }

        Queremos, portanto, maximizar $\frac{\text{custo unicast}}{\text{custo multicast}}$.
        Note-se que abordagens multicast são mais eficientes em topologias
        completamente esguias até ao último nível, já que não terá de haver divisão
        de datagramas por vários caminhos; por outro lado, e apesar de
        também ser mais eficiente do que abordagens binárias no caso de unicast,
        o caminho terá também de ser percorrido 1 vez por destino. Assim sendo, o
        cenário que maximiza o rácio pretendido é o seguinte:

        \begin{figure}[h]
          \centering
          \includesvg[width=0.4\textwidth]{assets/012b.svg}
        \end{figure}

        Em abordagens por emulação com encaminhamento unicast, o custo será, considerando
        $N$ níveis, $N \times D$; por outro lado, com encaminhamento multicast,
        o custo será $(N - 1) + D$. Considerando $D = 32$, ou seja, introduzindo
        o contexto do problema, a razão pretendida tenderá então para 32:

        $$
          \frac{32(N + 1)}{N - 1 + 32} \to 32
        $$

\end{enumerate}