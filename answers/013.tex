\question{
  \item Considere a rede da figura. O nó $D$ envia um datagrama que é difundido por
  toda a rede usando expedição por caminho inverso (RPF). Indique quantas cópias do
  datagrama é que atravessam as ligações da rede.

  \begin{figure}[H]
    \centering
    \includesvg[width=0.4\textwidth]{assets/013.svg}
  \end{figure}
}

A lógica do algoritmo em questão é relativamente simples: corresponde a uma
"variação" do algoritmo de Dijkstra, em que sempre que olhamos para um novo nó, fazemos
\textit{broadcast} do datagrama para todos os seus vizinhos (exceto o pai).

\begin{table}[H]
  \centering
  \begin{tabular}{l|l|l|l|l|l|l|l|l|l}
      & Nós Vistos                   & $d_A, \pi_A$ & $d_B, \pi_B$ & $d_C, \pi_C$ & $d_D, \pi_D$ & $d_E, \pi_E$ & $d_F, \pi_F$ & $d_G, \pi_G$ & $d_H, \pi_H$ \\ \hline
    1 & $\{D\}$                      & $\infty$     & 9            & 1            & 0            & 1            & 3            & 1            & $\infty$     \\
    2 & $\{D, C\}$                   & 5            & 3            & 1            & 0            & 1            & 3            & 1            & $\infty$     \\
    3 & $\{D, C, E\}$                & 5            & 3            & 1            & 0            & 1            & 2            & 1            & $\infty$     \\
    4 & $\{D, C, E, G\}$             & 5            & 3            & 1            & 0            & 1            & 2            & 1            & 15           \\
    5 & $\{D, C, E, G, F\}$          & 5            & 3            & 1            & 0            & 1            & 2            & 1            & 15           \\
    6 & $\{D, C, E, G, F, B\}$       & 4            & 3            & 1            & 0            & 1            & 2            & 1            & 5            \\
    7 & $\{D, C, E, G, F, B, A\}$    & 4            & 3            & 1            & 0            & 1            & 2            & 1            & 5            \\
    8 & $\{D, C, E, G, F, B, A, H\}$ & 4            & 3            & 1            & 0            & 1            & 2            & 1            & 5
  \end{tabular}
\end{table}

\begin{figure}[H]
  \centering
  \includesvg[width=0.4\textwidth]{assets/013a.svg}
\end{figure}
