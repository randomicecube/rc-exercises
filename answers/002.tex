\question{
  \item Os sistemas terminais $A$ e $B$ estão ligados a um mesmo comutador de pacotes.
  O sistema $A$ gera pacotes de dados com $480$ bytes cada a intervalos regulares de $T$ segundos,
  sendo o primeiro desses pacotes transmitido no instante $t = 0 ms$. A ligação de $A$ ao
  comutador tem débito $1$ $Mbps$. O sistema $B$ gera pacotes de voz codificada a $64$ $kbit/s$ a
  intervalos regulares de $20$ $ms$, sendo o primeiro desses pacotes transmitido no instante
  $t = 3 ms$. A ligação de $B$ ao comutador tem também débito $1$ $Mbps$. A linha de saída do
  comutador tem débito $256$ $kbit/s$. Desprezam-se os atrasos de processamento e de propagação.
}

\begin{enumerate}[leftmargin=\labelsep]
  \subquestion{
  \item Determine a gama de valores de $T$ para os quais o sistema é estável, i.e.,
        tal que o número de pacotes à espera de transmissão à saída do comutador não
        aumenta indefinidamente no tempo.
        }

        \subquestion{
  \item Assumindo que $T$ é tal que o sistema é estável e que a linha de saída do
        comutador transmite pacotes na ordem de chegada, independentemente se serem de
        dados ou de voz, diga qual o atraso máximo na transmissão dos pacotes de voz,
        medido desde o momento em que um bit está disponível à saída do codificador de voz
        até que é transmitido na linha de saída do comutador.
        }
\end{enumerate}