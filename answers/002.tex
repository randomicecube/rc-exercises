\question{
  \item Os sistemas terminais $A$ e $B$ estão ligados a um mesmo comutador de pacotes.
  O sistema $A$ gera pacotes de dados com $480$ bytes cada a intervalos regulares de $T$ segundos,
  sendo o primeiro desses pacotes transmitido no instante $t = 0 ms$. A ligação de $A$ ao
  comutador tem débito $1$ $Mbps$. O sistema $B$ gera pacotes de voz codificada a $64$ $kbit/s$ a
  intervalos regulares de $20$ $ms$, sendo o primeiro desses pacotes transmitido no instante
  $t = 3 ms$. A ligação de $B$ ao comutador tem também débito $1$ $Mbps$. A linha de saída do
  comutador tem débito $256$ $kbit/s$. Desprezam-se os atrasos de processamento e de propagação.
}

\begin{enumerate}[leftmargin=\labelsep]
  \subquestion{
  \item Determine a gama de valores de $T$ para os quais o sistema é estável, i.e.,
        tal que o número de pacotes à espera de transmissão à saída do comutador não
        aumenta indefinidamente no tempo.
        }

        O sistema é estável caso o ritmo a que os pacotes de dados chegam ao comutador seja
        menor ou igual ao ritmo a que os pacotes de dados são transmitidos na linha de saída
        do mesmo. Note-se que, tal como na alínea anterior, estamos a considerar
        um mecanismo de \textit{Store and Forward}, pelo que o pacote terá de ser
        completamente recebido no comutador antes de ser transmitido na linha de saída.
        Assim sendo:

        $$
          \text{output rate} \geq \text{input rate} \leftrightarrow
          256 \text{ kbit/s} \geq 64 \text{ kbit/s} + \text{rate de A}
        $$

        Neste sentido, o ritmo de saída de $A$ não poderá exceder $192$ $kbit/s$ -
        com pacotes de $480$ bytes, isto quer dizer que só devemos enviar pacotes
        de $A$ a intervalos de $T \geq \frac{480 \times 2^3}{192 \times 10^3} = 20ms$.

        \subquestion{
  \item Assumindo que $T$ é tal que o sistema é estável e que a linha de saída do
        comutador transmite pacotes na ordem de chegada, independentemente de serem de
        dados ou de voz, diga qual o atraso máximo na transmissão dos pacotes de voz,
        medido desde o momento em que um bit está disponível à saída do codificador de voz
        até que é transmitido na linha de saída do comutador.
        }

        \textbf{Nota: exercício atualmente errado, de acordo com as soluções.}

        Sendo o sistema estável, temos como pior cenário o caso em que os pacotes de dados
        e voz chegam ao mesmo tempo ao comutador, e em que o comutador transmite primeiro
        o pacote de dados. Neste caso, a quantidade de tempo pedida corresponde ao tempo
        de transmissão de um pacote de voz entre $B$ e o comutador, somado ao tempo de
        transmissão de um pacote de dados entre o comutador e o exterior.

        Em primeiro lugar, é importante notar que não é dado o tamanho de um pacote de voz.
        Sabemos, contudo, que é criado um pacote a cada $20 ms$, a um ritmo de $64$ $kbit/s$.
        Assim sendo, são criados $50$ pacotes por secundo, pelo que cada um deles terá
        $\frac{64 \times 10^3}{50} = 1280$ bits. O tempo de transmissão de um pacote de voz
        entre $B$ e o comutador é, portanto, dado por:

        $$
          t_{B-com} = 1.28 \times 10^{-3} \text{ s} = 1.28 \text{ ms}
        $$

        O tempo de transmissão de um pacote de dados entre o comutador e o exterior é dado
        pela seguinte expressão:

        $$
          t_{com-out} = \frac{480 \times 2^3}{256 \times 10^3} = 15 \text{ ms}
        $$

        Assim sendo, o atraso máximo na transmissão dos pacotes de voz é dado por $0.128 + 15 = 16.28$ $ms$.
\end{enumerate}