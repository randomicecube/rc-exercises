\question{
  \item O diagrama da figura mostra uma rede Wi-Fi em que $X$ é o ponto de acesso
  e $A$, $B$ e $C$ são estações a ele associadas. Os círculos grandes centrados em
  cada uma das estações representam a sua área de cobertura, isto é, o alcance do seu
  sinal eletromagnético. A área de cobertura do ponto de acesso X não está representada, mas
  subentende-se que cobre as três estações. O protocolo de acesso ao meio é CSMA/CA.
  Suponha que $X$ está a transmitir uma trama no instante 0 $\mu s$ que acabará de ser transmitida
  no instante $100 \mu s$. Suponha ainda o seguinte:

  \begin{table}[H]
    \centering
    \begin{tabular}{l|l|l|l}
      Estação                                                                                     &
      \begin{tabular}[c]{@{}l@{}}Instante em que tem\\ trama p/ transmitir ($\mu s$)\end{tabular} &
      \begin{tabular}[c]{@{}l@{}}Duração de transmissão\\ da trama ($\mu s$)\end{tabular}         &
      \begin{tabular}[c]{@{}l@{}}Tempo de recuo\\ (backoff) ($\mu s$)\end{tabular}                                       \\ \hline
      $A$                                                                                         & $50$ & $100$ & $70$  \\
      $B$                                                                                         & $70$ & $200$ & $200$ \\
      $C$                                                                                         & $90$ & $150$ & $150$
    \end{tabular}
    \label{tab:ex-17}
  \end{table}

  \begin{figure}[H]
    \centering
    \includesvg[width=0.3\textwidth]{assets/017.svg}
  \end{figure}
}

O seguinte diagrama espaço-tempo será útil para a resolução das alíneas que não
contemplam CSMA/CA com RTS-CTS:

\begin{figure}[H]
  \centering
  \includesvg[width=0.9\textwidth]{assets/017-empty-csma.svg}
\end{figure}

Tal como se pode visualizar no diagrama acima, enquanto $X$ tenta transmitir,
todas as estações tentam transmitir (mas não conseguem, porque todas sabem
que $X$ está também a transmitir). Neste sentido, após essa transmissão,
todas as estações entram num período de \textit{back-off}, sendo que a primeira
a sair dele é a estação $A$, que começa a transmitir a sua trama no instante
$170 \mu s$. Ora, $B$ não consegue "ouvir" $A$, pelo que não interrompe o
seu período de back-off (ao contrário de $C$, que interrompe passados $70 \mu s$,
restando $80 \mu s$ para o seu período terminar). Assim, quando $A$ termina
de transmitir, faltam $30 \mu s$ para $B$ terminar o seu período de back-off,
e $80 \mu s$ para $C$ terminar o seu. $B$ começa a transmitir a sua trama
aos $300 \mu s$, e $C$ aos $350 \mu s$, entrando em colisão (já que nenhum dos
dois ouve o outro). No fim, $X$ não faz \textit{acknowledgement} de nenhuma
das tramas, pelo que só agora é que as estações se apercebem da colisão.

\begin{enumerate}[leftmargin=\labelsep]
  \subquestion{
  \item Para cada uma das três estações, em que instante de tempo é que ela começa
        a transmitir a sua trama pela primeira vez? Despreze atrasos de propagação,
        intervalos-entre-tramas (inter-frame spacings) e tempos de transmissão dos ACK.
        }

        Tal como referido na explicação acima:

        \begin{itemize}
          \item $A$: $170 \mu s$
          \item $B$: $300 \mu s$
          \item $C$: $350 \mu s$
        \end{itemize}

        \subquestion{
  \item Das três tramas transmitidas pelas estações quais é que são bem recebidas
        na primeira tentativa de transmissão?
        }

        Apenas a trama de $A$ é bem recebida na primeira tentativa de transmissão,
        tendo o respetivo ACK sido enviado por $X$ no instante $270 \mu s$.

        \subquestion{
  \item Para cada uma das tramas que não é bem recebida na primeira tentativa de transmissão,
        indique qual o instante de tempo em que a estação correspondente se apercebe de que
        a trama foi perdida. As tramas serão bem recebidas na segunda tentativa de transmissão?
        }

        Ambas as estações apercebem-se no mesmo instante, $500 \mu s$. Nenhuma delas
        conseguirá transmitir sem colisão na segunda tentativa, já que $B$ voltará a transmitir
        enquanto $C$ transmite.

        \subquestion{
  \item Suponha agora que se ativa o protocolo de acesso ao meio RTS-CTS. Neste caso, em
        que instante de tempo é que cada estação começa a transmitir a sua trama pela
        primeira vez? As tramas são todas recebidas com sucesso?
        }

        O diagrama espaço-tempo para esta interação é o seguinte:

        \begin{figure}[H]
          \centering
          \includesvg[width=0.9\textwidth]{assets/017-rts-cts.svg}
        \end{figure}

        Note-se que ao utilizar este protocolo de acesso ao meio, não há obrigatoriedade
        das várias estações se ouvirem, já que todas esperam pela concessão de
        acesso por parte de $X$. Temos, claro, que agora todas tramas são recebidas com
        sucesso, nos instantes indicados na figura.
\end{enumerate}