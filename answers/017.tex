\question{
  \item O diagrama da figura mostra uma rede Wi-Fi em que $X$ é o ponto de acesso
  e $A$, $B$ e $C$ são estações a ele associadas. Os círculos grandes centrados em
  cada uma das estações representam a sua área de cobertura, isto é, o alcance do seu
  sinal eletromagnético. A área de cobertura do ponto de acesso X não está representada, mas
  subentende-se que cobre as três estações. O protocolo de acesso ao meio é CSMA/CA.
  Suponha que $X$ está a transmitir uma trama no instante 0 $\mu s$ que acabará de ser transmitida
  no instante $100 \mu s$. Suponha ainda o seguinte:

  \begin{table}[H]
    \centering
    \begin{tabular}{l|l|l|l}
      Estação                                                                                     &
      \begin{tabular}[c]{@{}l@{}}Instante em que tem\\ trama p/ transmitir ($\mu s$)\end{tabular} &
      \begin{tabular}[c]{@{}l@{}}Duração de transmissão\\ da trama ($\mu s$)\end{tabular}         &
      \begin{tabular}[c]{@{}l@{}}Tempo de recuo\\ (backoff) ($\mu s$)\end{tabular}                                       \\ \hline
      $A$                                                                                         & $50$ & $100$ & $70$  \\
      $B$                                                                                         & $70$ & $200$ & $200$ \\
      $C$                                                                                         & $90$ & $150$ & $150$
    \end{tabular}
    \label{tab:ex-17}
  \end{table}

  \begin{figure}[H]
    \centering
    \includesvg[width=0.4\textwidth]{assets/017.svg}
  \end{figure}
}

\begin{enumerate}[leftmargin=\labelsep]
  \subquestion{
  \item Para cada uma das três estações, em que instante de tempo é que ela começa
        a transmitir a sua trama pela primeira vez? Despreze atrasos de propagação,
        intervalos-entre-tramas (inter-frame spacings) e tempos de transmissão dos ACK.
        }
        \subquestion{
  \item Das três tramas transmitidas pelas estações quais é que são bem recebidas
        na primeira tentativa de transmissão?
        }
        \subquestion{
  \item Para cada uma das tramas que não é bem recebida na primeira tentativa de transmissão,
        indique qual o instante de tempo em que a estação correspondente se apercebe de que
        a trama foi perdida. As tramas serão bem recebidas na segunda tentativa de transmissão?
        }
        \subquestion{
  \item Suponha agora que se ativa o protocolo de acesso ao meio RTS-CTS. Neste caso, em
        que instante de tempo é que cada estação começa a transmitir a sua trama pela
        primeira vez? As tramas são todas recebidas com sucesso?
        }
\end{enumerate}