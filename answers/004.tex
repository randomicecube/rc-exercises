\question{
  \item Considere as redes da figura, na qual Serv. DNS local é o servidor de nomes
  configurado em \texttt{me.y.com} e Serv. DNS \texttt{x.com} é o servidor de nomes idóneo (Authoritative
  Name Server) para o domínio \texttt{x.com}. O tempo de propagação a separar um encaminhador
  do outro é $100$ $ms$ e o tempo de propagação entre as máquinas da rede \texttt{y.com} e a
  Internet é de $500$ $ms$ (num sentido apenas). Inicialmente, o Serv. DNS local tem a
  sua cache de nomes vazia.

  \begin{figure}[H]
    \centering
    \includesvg[width=0.4\textwidth]{assets/004.svg}
  \end{figure}
}

\begin{enumerate}[leftmargin=\labelsep]
  \subquestion{
  \item Suponha que o utilizador de um browser em \texttt{me.y.com} digita a URL \texttt{www.x.com/index.html}.
        Apresente a sequência ordenada de todas as mensagens DNS e HTTP trocadas
        até que o ficheiro \texttt{index.html} seja recebido em \texttt{me.y.com}.
        Relembra-se que a pesquisa de nomes a partir de Serv. DNS local é iterativa.
        Para cada mensagem indique a sua origem, o seu destino e o tipo de mensagem.
        (Seja tão preciso quanto possível relativamente ao tipo da mensagem).
        }

        A tabela seguinte corresponde a uma listagem ordenada das mensagens DNS e HTTP trocadas.

        \begin{table}[H]
          \centering
          \begin{tabular}{l|l|l|l|l}
               & \textbf{Tipo} & \textbf{Origem}  & \textbf{Destino} & \textbf{Comentário}                        \\
            1  & DNS Query     & me.y.com         & Serv. DNS local  & -                                          \\
            2  & DNS Query     & Serv. DNS local  & Root Name Server & -                                          \\
            3  & DNS Response  & Root Name Server & Serv. DNS local  & Root não sabe, aponta p/ TLD               \\
            4  & DNS Query     & Serv. DNS local  & TLD Server       & -                                          \\
            5  & DNS Response  & TLD Server       & Serv. DNS local  & TLD não sabe, aponta p/ Serv. DNS de x.com \\
            6  & DNS Query     & Serv. DNS local  & Serv. DNS x.com  & -                                          \\
            7  & DNS Response  & Serv. DNS x.com  & Serv. DNS local  & Serv. DNS de x.com indica o IP da máquina  \\
            8  & DNS Response  & Serv. DNS local  & me.y.com         & me.y.com terá agora o IP de x.com          \\
            9  & GET HTTP/1.1  & me.y.com         & www.x.com        & Pedido index.html                          \\
            10 & HTTP 200 OK   & www.x.com        & me.y.com         & Ficheiro enviado
          \end{tabular}
        \end{table}

        \subquestion{
  \item Sabendo que o ficheiro \texttt{index.html} residente na máquina \texttt{www.x.com} tem 1 Mbit,
        determine o atraso desde o momento em que o utilizador digita a URL em \texttt{me.y.com}
        até que o ficheiro é recebido na totalidade. Considere que o mecanismo de
        arranque lento não está ativo e despreze o tempo de transmissão de todos os
        segmentos exceto aqueles que contêm dados do ficheiro \texttt{index.html}.
        }

        Para calcular o atraso pretendido, faz sentido suportar o nosso pensamento na tabela
        anterior. Note-se que o tempo de propagação entre \texttt{me.y.com} e o
        respetivo servidor de nomes é considerado desprezável.

        \begin{itemize}
          \item As mensagens 2 e 3, entre Serv. DNS local e Root Name Server,
                levam $2 \cdot 500$ $ms$, já que o tempo de propagação num só sentido
                é de $500$ $ms$.
          \item As mensagens 4 e 5, entre Serv. DNS local e TLD Server,
                levam $2 \cdot 500$ $ms$.
          \item As mensagens 6 e 7, entre Serv. DNS local e Serv. DNS x.com,
                levam $2 \cdot 100$ $ms$, já que aqui não vamos à \textit{black-box} Internet,
                passando diretamente entre comutadores.
          \item As mensagens 9 e 10 levam $2 \cdot 2 \cdot 100$ $ms$, contando com o par
                SYN/ACK e o par ACK GET/transmissão do ficheiro.
        \end{itemize}

        Assim sendo, o atraso total é de $2 \cdot 500 + 2 \cdot 500 + 2 \cdot 100 + 2 \cdot 2 \cdot 100 + 1000 = 3600$ $ms$.

        \subquestion{
  \item Assuma agora que:
        (i) o ficheiro \texttt{index.html} referencia uma imagem com $5$ $Mbits$;
        (ii) o browser usa sessões HTTP persistentes;
        (iii) o mecanismo de arranque lento está ativo;
        (iv) e o MSS é $50$ $kbit$.
        Determine o atraso na receção da página, constituída pelo ficheiro mais a imagem.
        }

        Em primeiro lugar, é relevante realçar que é mencionada a existência de sessões
        HTTP persistentes, mas nada é referido no que a pipelining diz respeito. Neste sentido,
        essa noção será aqui descartada. Mais ainda, é referido o mecanismo de arranque lento,
        com MSS de $50$ $kbit$. Note-se que esta alteração não afeta os tempos de propagação até
        ao envio do primeiro segmento de \textit{index.html} (inclusive), pelo que a \textit{baseline}
        temporal será de $2600$ $ms$.

        As mensagens seguintes verão um diagrama espaço-tempo como o que se segue:

        \begin{figure}[H]
          \centering
          \includesvg[width=0.4\textwidth]{assets/004c.svg}
        \end{figure}

        Assim sendo, aos $2600$ $ms$ da \textit{baseline} acrescem os seguintes tempos:

        \begin{itemize}
          \item 0.05 segundos para a transferência do primeiro segmento;
          \item 0.2 + 0.1 segundos para a transferência dos dois segmentos seguintes;
          \item 0.2 + $x$ segundos para a transferência dos restantes segmentos - aqui, após
                o arranque lento, estamos finalmente num cenário de transferência de dados
                contínuo. Tendo já sido transferidos três segmentos, resta-nos transferir $N-3$ segmentos,
                onde $N$ corresponde ao número de segmentos necessários para transmitir
                o ficheiro \texttt{html} e respetiva imagem.
        \end{itemize}


        $$
          N = \frac{5 \cdot 10^6 + 1 \cdot 10^6}{50 \cdot 10^3} = 120
        $$

        Restando transferir $N-3 = 117$ segmentos, temos que o tempo de transferência dos restantes
        segmentos será dado por $x = 0.05 \cdot 117 = 5.85$ segundos. Note-se que será ainda
        adicionado um RTT, $200$ $ms$, para a transferência do ficheiro de imagem.

        Assim, temos que no total, a interação em questão leva $2600 + 50 + 200 + 100 + 5850 + 200 = 9000$ $ms$.

\end{enumerate}