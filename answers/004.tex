\question{
  \item Considere as redes da figura, na qual Serv. DNS local é o servidor de nomes
  configurado em \texttt{me.y.com} e Serv. DNS \texttt{x.com} é o servidor de nomes idóneo (Authoritative
  Name Server) para o domínio \texttt{x.com}. O tempo de propagação a separar um encaminhador
  do outro é $100$ $ms$ e o tempo de propagação entre as máquinas da rede \texttt{y.com} e a
  Internet é de $500$ $ms$ (num sentido apenas). Inicialmente, o Serv. DNS local tem a
  sua cache de nomes vazia.

  \begin{figure}[H]
    \centering
    \includesvg[width=0.5\textwidth]{assets/004.svg}
  \end{figure}
}

\begin{enumerate}[leftmargin=\labelsep]
  \subquestion{
  \item Suponha que o utilizador de um browser em \texttt{me.y.com} digita a URL \texttt{www.x.com/index.html}.
        Apresente a sequência ordenada de todas as mensagens DNS e HTTP trocadas
        até que o ficheiro \texttt{index.html} seja recebido em \texttt{me.y.com}.
        Relembra-se que a pesquisa de nomes a partir de Serv. DNS local é iterativa.
        Para cada mensagem indique a sua origem, o seu destino e o tipo de mensagem.
        (Seja tão preciso quanto possível relativamente ao tipo da mensagem).
        }

        \subquestion{
  \item Sabendo que o ficheiro \texttt{index.html} residente na máquina \texttt{www.x.com} tem 1 Mbit,
        determine o atraso desde o momento em que o utilizador digita a URL em \texttt{me.y.com}
        até que o ficheiro é recebido na totalidade. Considere que o mecanismo de
        arranque lento não está ativo e despreze o tempo de transmissão de todos os
        segmentos exceto aqueles que contêm dados do ficheiro \texttt{index.html}.
        }

        \subquestion{
  \item Assuma agora que:
        (i) o ficheiro \texttt{index.html} referencia uma imagem com $5$ $Mbits$;
        (ii) o browser usa sessões HTTP persistentes;
        (iii) o mecanismo de arranque lento está ativo;
        (iv) e o MSS é $50$ $kbit$.
        Determine o atraso na receção da página, constituída pelo ficheiro mais a imagem.
        }
\end{enumerate}