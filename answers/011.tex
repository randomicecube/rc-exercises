\question{
  \item Considere a rede da figura sobre a qual opera um protocolo vetor-distância.

  \begin{figure}[H]
    \centering
    \includesvg[width=0.3\textwidth]{assets/011.svg}
  \end{figure}
}

\begin{enumerate}[leftmargin=\labelsep]
  \subquestion{
  \item No instante $t_0$ o protocolo está estável, com todos os nós a saber os
        comprimentos dos caminhos mais curtos para alcançar cada um dos outros nós. Apresente
        as entradas das tabelas de encaminhamento no que diz respeito ao nó destino $D$.
        }

        No instante inicial:

        \begin{itemize}
          \item $A$ sabe que o caminho mais curto para $D$ é direto, com custo 1.
          \item $B$ sabe que o caminho mais curto para $D$ é $B-A-D$, com custo 3.
          \item $C$ sabe que o caminho mais curto para $D$ é $C-A-D$, com custo 2.
          \item $D$ para $D$ é direto, com custo 0.
        \end{itemize}

        \subquestion{
  \item No instante $t_1$ a ligação $A-D$ falha. Assumindo que os nós trocam mensagens
        sincronamente em instantes bem definidos, $t_1, t_2, t_3, ...$, mostre a evolução das
        tabelas de encaminhamento até que o protocolo volte a estabilizar.
        }

        Inicialmente, todas as tabelas de custo tinham o seguinte aspeto (para todos os nós):

        \begin{table}[H]
          \centering
          \begin{tabular}{l|l|l|l|l}
            From/Cost to & A & B & C & D \\ \hline
            A            & 0 & 2 & 1 & 1 \\
            B            & 2 & 0 & 2 & 3 \\
            C            & 1 & 2 & 0 & 2 \\
            D            & 1 & 3 & 2 & 0
          \end{tabular}
        \end{table}

        Após a ligação $A-D$ falhar, no instante $t_1$, as tabelas de $A$ e $D$
        são atualizadas:

        \begin{table}[H]
          \centering
          \begin{tabular}{l|l|l|l|l}
            From/Cost to & A & B & C & D                          \\ \hline
            A            & 0 & 2 & 1 & $\min\{2 + 3, 1 + 2\} = 3$ \\
            B            & 2 & 0 & 2 & 3                          \\
            C            & 1 & 2 & 0 & 2                          \\
            D            & 1 & 3 & 2 & $\min\{3 + 2, 2 + 1\} = 3$
          \end{tabular}
        \end{table}

        \dots (acabo depois)

        \subquestion{
  \item Repita a alínea anterior, mas assumindo agora que o protocolo utilizado é
        vetor-caminho. Neste protocolo os nós trocam entre si não apenas a distância para
        alcançar cada destino mas também todo o caminho (sequência de nós) associado a essa distância.
        }
\end{enumerate}