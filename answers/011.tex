\question{
  \item Considere a rede da figura sobre a qual opera um protocolo vetor-distância.

  \begin{figure}[H]
    \centering
    \includesvg[width=0.4\textwidth]{assets/011.svg}
  \end{figure}
}

\begin{enumerate}[leftmargin=\labelsep]
  \subquestion{
  \item No instante $t_0$ o protocolo está estável, com todos os nós a saber os
        comprimentos dos caminhos mais curtos para alcançar cada um dos outros nós. Apresente
        as entradas das tabelas de encaminhamento no que diz respeito ao nó destino $D$.
        }
        \subquestion{
  \item No instante $t_1$ a ligação $A-D$ falha. Assumindo que os nós trocam mensagens
        sincronamente em instantes bem definidos, $t_1, t_2, t_3, ...$, mostre a evolução das
        tabelas de encaminhamento até que o protocolo volte a estabilizar.
        }
        \subquestion{
  \item Repita a alínea anterior, mas assumindo agora que o protocolo utilizado é
        vetor-caminho. Neste protocolo os nós trocam entre si não apenas a distância para
        alcançar cada destino mas também todo o caminho (sequência de nós) associado a essa distância.
        }
\end{enumerate}