\question{
  \item Suponha que um segmento TCP tem $2048$ bytes de dados e $20$ bytes de cabeçalho.
  Este segmento tem que atravessar duas ligações para chegar ao destino. A primeira
  ligação tem um MTU de $1024$ bytes e a segunda um MTU de $512$ bytes. Indique o
  comprimento e offset de todos os fragmentos entregues à camada de rede do destino.
  O cabeçalho de qualquer datagrama IP tem $20$ bytes.
}

A resolução deste exercício tem algumas particularidades que o tornam menos trivial.
O primeiro pensamento seria provavelmente dividir o segmento TCP (com 2068 bytes)
em três datagramas IP, dois com $1004 + 20$ bytes e um com $2 \times 20 + 20 + 20$ bytes.
Ora, isto não é possível, já que juntamente com o datagrama IP é também enviado um
offset ($\frac{\text{data length}}{8}$) inteiro. Neste caso, $\frac{1004}{8}$ não é um inteiro,
pelo que seremos obrigados a enviar, no máximo $1000$ bytes de dados. Assim sendo,
de TCP para IP ficamos com três segmentos:

\begin{itemize}
  \item Dois segmentos de tamanho $1020$ bytes, com $1000$ bytes de dados;
  \item Um segmento com os restantes $68$ bytes de dados (isto inclui os 20 bytes
        do cabeçalho TCP), mais $20$ bytes de cabeçalho. Note-se que, sendo o último elemento,
        a necessidade de \text{data length} divisível por 8 não se aplica.
\end{itemize}

Por fim, podemos então olhar para a segunda ligação IP. Como o MTU é de $512$ bytes,
em cada datagrama IP podemos enviar $492$ bytes de dados. Mais uma vez, $492$ não
é divisível por 8 - na verdade, podemos enviar no máximo, por segmento, $488$ bytes
de dados. Assim sendo, temos:

\begin{table}[H]
  \centering
  \begin{tabular}{l|l|l|l}
      & Datagram size & Data size & Offset (8-byte blocks) \\ \hline
    1 & 508           & 488       & 0                      \\
    2 & 508           & 488       & 61                     \\
    3 & 44            & 24        & 122                    \\
    4 & 508           & 488       & 125                    \\
    5 & 508           & 488       & 186                    \\
    6 & 44            & 24        & 247                    \\
    7 & 88            & 68        & 250
  \end{tabular}
\end{table}

Note-se que o offset vai sendo incrementando em $\frac{\text{data size}}{8}$ blocos
por datagrama, claro. Mais ainda, notar que o último datagrama não teve necessidade
de ser truncado, pelo que a sua estrutura manteve-se intacta.
