\question{
  \item Considere três estações que pretendem recuperar um ficheiro com $100$ $MiB$ de um servidor.
  A largura-de banda uplink do servidor, do servidor à Internet, é de $1$ $Mbit/s$. A largura
  de banda downlink de cada estação, da Internet à estação, é muito elevada (superior a
  $1$ $Mbit/s$) e a largura de banda uplink de cada estação, da estação à Internet, é $200$ $kbit/s$.
  Despreze os atrasos de propagação.
}

\begin{enumerate}[leftmargin=\labelsep]
  \subquestion{
  \item Com uma aplicação cliente servidor, qual o tempo mínimo necessário para que
        o ficheiro seja distribuído a todas as estações?
        }

        \subquestion{
  \item Considere agora uma aplicação peer-to-peer em que as estações usam as suas
        larguras de banda uplink para ajudar a distribuir o ficheiro. O servidor envia
        dados a cada uma das estações continuamente. Para além disso, cada estação vai
        redistribuir a cada uma das outras duas estações os dados que recebe do servidor.
        Encontre uma boa estratégia para distribuir o ficheiro por todas as estações, e
        de acordo com ela diga qual o tempo mínimo necessário para a distribuição do ficheiro?
        }
\end{enumerate}