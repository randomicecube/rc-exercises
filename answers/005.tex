\question{
  \item Considere três estações que pretendem recuperar um ficheiro com $100$ $MiB$ de um servidor.
  A largura-de banda uplink do servidor, do servidor à Internet, é de $1$ $Mbit/s$. A largura
  de banda downlink de cada estação, da Internet à estação, é muito elevada (superior a
  $1$ $Mbit/s$) e a largura de banda uplink de cada estação, da estação à Internet, é $200$ $kbit/s$.
  Despreze os atrasos de propagação.
}

\begin{enumerate}[leftmargin=\labelsep]
  \subquestion{
  \item Com uma aplicação cliente servidor, qual o tempo mínimo necessário para que
        o ficheiro seja distribuído a todas as estações?
        }

        O servidor terá de distribuir o envio do ficheiro por três clientes, pelo
        que a largura de banda uplink do mesmo será dividida por três. Assim sendo,
        e considerando que a largura de banda downlink de cada estação é muito elevada (pelo que este não será o \textit{bottleneck}),
        o tempo mínimo necessário para que o ficheiro seja distribuído a todas as estações
        é dado por:

        $$
          T_{min} = 3 \times \frac{100 \text{ MiB}}{1 \text{ Mbit/s}} = 3 \times \frac{100 \times 2^{20} \times 2^3}{1 \times 10^6} = 2516.5824 \text{ s}
        $$

        \subquestion{
  \item Considere agora uma aplicação peer-to-peer em que as estações usam as suas
        larguras de banda uplink para ajudar a distribuir o ficheiro. O servidor envia
        dados a cada uma das estações continuamente. Para além disso, cada estação vai
        redistribuir a cada uma das outras duas estações os dados que recebe do servidor.
        Encontre uma boa estratégia para distribuir o ficheiro por todas as estações, e
        de acordo com ela diga qual o tempo mínimo necessário para a distribuição do ficheiro?
        }

        O ficheiro é distribuído por todas as estações, sendo dividido em $N$ (com $N=3$)
        partes distintas, uma por estação. Cada estação recebe uma parte do ficheiro e
        envia a cada uma das outras duas estações a sua parte. Assim sendo, a largura de
        banda uplink de cada estação será dividida por $N-1$. Assim sendo:

        $$
          T_{upload-servidor} = 3 \times \frac{\frac{100}{3} \times 2^{23}}{1 \times 10^6} = 838.8608 \text{ s}
        $$

        Resta ainda calcular o tempo de upload de cada estação para as outras duas estações.

        $$
          T_{upload-estacao} = \frac{N \times L}{u_{servidor} + \sum{u_{estacao}}} = \frac {3 \times 100 \times 2^{23}}{10^6 + 3 \times 200 \times 10^3} = 1572.864 \text{ s}
        $$

        Note-se que o tempo mínimo necessário para a distribuição do ficheiro é dado
        (considerando que o \textit{download} não é o \textit{bottleneck}) por:

        $$
          T_{min} = \max\{T_{upload-servidor}, T_{upload-estacao}\} = 1572.864 \text{ s}
        $$
\end{enumerate}