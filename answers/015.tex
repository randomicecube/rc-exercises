\question{
  \item Suponha duas estações, $A$ e $B$, ligadas a extremos opostos de um cabo de
  $900 m$, e cada uma delas com uma trama de 1000 bits para transmitir. As duas estações
  começam a transmitir em $t = 0$. Suponha que há $4$ repetidores entre $A$ e $B$,
  cada um inserindo um atraso de $20$ bits. Considere que a taxa de transmissão é
  $10 Mbit/s$ e que se utiliza CSMA/CD com intervalos de retransmissão múltiplos de $512$ bits.
  Depois da primeira colisão, $A$ escolhe $K_A = 0$ e $B$ escolhe $K_B = 1$ na execução do
  algoritmo de recuo binário exponencial. Ignore o sinal de reforço de colisão de
  $32$ bits e a espera de $96$ bits para o acesso ao canal.
}

\begin{enumerate}[leftmargin=\labelsep]
  \subquestion{
  \item Qual o atraso de propagação, incluindo atrasos introduzidos nos repetidores
        de $A$ para $B$? Suponha uma velocidade de propagação no cabo de $2 \times 10^8 m/s$.
        }

        O atraso de propagação pretendido corresponderá à soma entre o tempo de propagação
        normal, sem repetidores, e o tempo de propagação introduzido pelos mesmos.

        No cabo, a velocidade de propagação é  de $2 \times 10^8 m/s$, pelo que, percorrendo
        $900 m$, o tempo de propagação será de $4.5 \times 10^{-6} s$. Mais ainda, é introduzido
        um atraso de $20$ bits em cada repetidor - podendo transmitir $10 Mbit/s = 10^7 bit/s$, cada repetidor
        introduzirá, assim, um atraso de $2 \times 10^{-6} s$. Assim sendo, no total,
        o atraso de propagação será de $4.5 \times 10^{-6} s + 4 \times 2 \times 10^{-6} s = 1.25 \times 10^{-5} s$.

        \subquestion{
  \item Em que instante de tempo é que o pacote de $A$ é completamente recebido em $B$?
        }

        O diagrama espaço-tempo da interação em questão é o que se segue:

        \begin{figure}[H]
          \centering
          \includesvg[width=0.6\textwidth]{assets/015b.svg}
        \end{figure}

        Passados $1.25 \times 10^{-6} s$ desde o início da transmissão, $A$ e $B$
        apercebem-se da colisão, pelo que param de transmitir - a transmissão em
        si só chega completamente ao outro lado $1.25 \times 10^{-6} s$ depois,
        pelo que só aqui já se passaram $2.5 \times 10^{-6} s$ desde o início da
        transmissão. De seguida, com $A$ a escolher $K=0$, pode logo voltar a
        transmitir; por outro lado, $B$ tem de esperar $512$ bits, o que corresponde
        a $5.12 \times 10^{-5}s$. Note-se que a transmissão de $1000$ bits leva
        $100 \times 10^{-6} s$, pelo que $B$ até tenta começar a transmitir sensivelmente
        a meio da transmissão de $A$, mas apercebendo-se que este está a transmitir,
        não o faz, voltando a tentar depois de $5.12 \times 10^{-5}s$ (e aqui a
        transmissão de $A$ já terminou). Ao todo, passaram $3 \times 12.5 \times 10^{-6} + 100 \times 10^{-6} = 137.5 \mu s$
        desde o início da transmissão, até ao pacote de $A$ ser completamente
        recebido em $B$.

\end{enumerate}