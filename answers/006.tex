\question{
  \item Considere um caminho de $5000$ $km$ de comprimento com atraso de propagação
  igual a $5$ $\mu s/km$, e sobre o qual consegue transmitir a um débito máximo de 100 Mbit/s.
  Suponha que usa um algoritmo de janela deslizante para controlo de erros e controlo de fluxo.
  Cada pacote tem 1000 bits. Se usar Go-Back-N qual a dimensão mínima da janela,
  em número de pacotes, que garante uma eficiência de utilização do caminho de 100\%?
  E se usar Selective-Repeat?
}

Garantimos eficiência de utilização de 100\% caso nunca se esteja \textit{idle} - isto é,
estamos sempre a transmitir ou receber dados. Para tal, é necessário que a janela seja
grande o suficiente para que, assim que o último pacote de um conjunto de $N$ pacotes
é enviado, começamos a receber o primeiro pacote desse mesmo conjunto.

Ora, em primeiro lugar será relevante calcular o RTT: neste caso, corresponde a
duas vezes o atraso de propagação sobre todo o comprimento do caminho, ou seja,
$2 \times 5000 \times 5 \times 10^{-6} = 0.05$ $s$.

Ora, o nosso objetivo é o de encontrar um valor $N_{window}$ tal que enviar $N_{window}$
pacotes seja \textbf{igual} (já que queremos eficiência de utilização de 100\%) ao tempo
que leva a enviar um pacote e o respetivo ACK chegar. Neste sentido, o que vamos querer
é o seguinte:

$$
  N_{window} \times \frac{L}{R} = \frac{L}{R} + RTT
$$

Assim sendo, isto será o mesmo que ter o seguinte:

$$
  N_{window} = 1 + \frac{RTT}{\frac{L}{R}} = 1 + \frac{0.05}{\frac{1000}{100 \times 10^6}} = 1 + 5000 = 5001
$$

A lógica é em tudo igual para o Selective-Repeat, já que o mecanismo aqui relevante
é o de janela deslizante, não sendo relevante a "tipologia" do mesmo.
