\question{
  \item Considere os fornecedores de serviços Internet (ISPs) $P$, $Q$ e $R$ interligados
  da forma ilustrada na figura e em que nenhum deles deverá transportar tráfego de
  trânsito entre os outros dois. Aos ISPs $P$, $Q$ e $R$ foram atribuídos os blocos de
  endereços \texttt{193.32.0.0/11}, \texttt{193.64.0.0/11} e \texttt{193.0.0.0/12}, respetivamente.
  As regiões autónomas (ASes) $P_1$ e $P_2$ são clientes do ISP $P$ e foram-lhes atribuídos
  os blocos de endereços \texttt{193.50.0.0/16} e \texttt{193.49.0.0/16}, respetivamente.
  As ASes $Q_1$ e $Q_2$ são clientes do ISP $Q$. A $Q_1$ foi atribuído o bloco de
  endereços \texttt{193.64.0.0/18}.

  \begin{figure}[H]
    \centering
    \includesvg[width=0.6\textwidth]{assets/008.svg}
  \end{figure}
}

\begin{enumerate}[leftmargin=\labelsep]
  \subquestion{
  \item Atribua um bloco de endereços a $Q_2$, sabendo que esta AS precisa de $8000$
        endereços e escolhendo o endereço base mais baixo possível.
        }
        \subquestion{
  \item Apresente as tabelas de expedição de um encaminhador em $P$, outro em $Q$,
        e um terceiro em $R$, com pares (prefixo, AS).
        }
        \subquestion{
  \item Suponha que $Q_1$ mudou o seu ISP de $Q$ para $P$. Apresente as novas tabelas
        de expedição tendo em conta que os endereços atribuídos a Q1 não se alteram.
        }

        \subquestion{
  \item Relativamente à alínea anterior, apresente em pseudocódigo um algoritmo para
        a expedição de datagramas usado num encaminhador do ISP $R$.
        }
\end{enumerate}