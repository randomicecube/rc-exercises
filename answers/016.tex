\question{
  \item Considere a rede da figura, composta por cinco comutadores Ethernet (Bridges):
  $V$, $W$, $X$, $Y$, $Z$, um hub e quatro estações: $A$, $B$, $C$, $D$. Considera-se que o identificador
  de cada comutador é igual ao menor dos endereços MAC das suas interfaces (indicados
  simplificadamente junto a cada ligação) - por exemplo, o identificador do comutador
  $W$ será $10$. Todas as ligações têm custo unitário. As tabelas estão inicialmente vazias.

  \begin{figure}[H]
    \centering
    \includesvg[width=\textwidth]{assets/016.svg}
  \end{figure}
}

\begin{enumerate}[leftmargin=\labelsep]
  \subquestion{
  \item Usando o algoritmo spanning tree classifique as interfaces de cada um dos
        comutadores em raiz, designada, ou bloqueada, e indique as BPDUs enviadas por cada
        comutador em cada uma das suas interfaces quando em regime estacionário.
        }
        \subquestion{
  \item Para a sequência de envio de tramas: ($A \to B$, $C \to D$, $C \to A$, $B \to A$, $D \to C$),
        indique as interfaces sobre as quais são transmitidas cópias das tramas respectivas
        e qual o estado das tabelas de expedição de cada comutador no final das várias transmissões.
        }
\end{enumerate}