\question{
  \item Considere a rede da figura em que o protocolo de encaminhamento usado é por estado-da-ligação.

  \begin{figure}[H]
    \centering
    \includesvg[width=0.5\textwidth]{assets/010.svg}
  \end{figure}
}

\begin{enumerate}[leftmargin=\labelsep]
  \subquestion{
  \item Preencha uma tabela ilustrando a execução do algoritmo de Dijkstra a partir do nó $A$.
        }

        (A aplicação do algoritmo já foi abordada \href{https://resumos.leic.pt/asa/caminhos-mais-curtos/#algoritmo-de-dijkstra}{em ASA}, pelo que
        não será descrita em pormenor nesta resposta.)

        \begin{table}[H]
          \centering
          \begin{tabular}{l|l|l|l|l|l|l}
              & Nós vistos          & $d_A, \pi_A$ & $d_B, \pi_B$ & $d_C, \pi_C$ & $d_D, \pi_D$ & $d_E, \pi_E$ \\ \hline
            1 & $\{A\}$             & 0, -         & 1, A         & 6, A         & 2, A         & $\infty$     \\
            2 & $\{A, B\}$          & 0, -         & 1, A         & 4, B         & 2, A         & $\infty$     \\
            3 & $\{A, B, D\}$       & 0, -         & 1, A         & 4, B         & 2, A         & 4, D         \\
            4 & $\{A, B, D, C\}$    & 0, -         & 1, A         & 4, B         & 2, A         & 4, D         \\
            5 & $\{A, B, D, C, E\}$ & 0, -         & 1, A         & 4, B         & 2, A         & 4, D
          \end{tabular}
        \end{table}

        \subquestion{
  \item Apresente o pseudocódigo genérico que cada nó tem que executar para a partir
        dos cálculos efetuados com o algoritmo de Dijkstra popular a sua tabela de expedição.
        Preencha a tabela de expedição do nó $A$.
        }

        O pseudocódigo é relativamente simples: para cada nó, após a aplicação de
        Dijkstra, temos o respetivo pai; basta então percorrer a árvore de caminhos
        mais curtos a partir do nó de destino, até ao nó de origem, por forma a
        obter o nó que a ele vai dar desde a origem - na tabela de expedição, o
        custo será o calculado pelo algoritmo de Dijkstra, e o nó de encaminhamento
        será o nó ligado à origem na árvore de caminhos mais curtos que liga o
        nó de destino ao nó de origem. Para o caso da alínea anterior, teríamos
        uma árvore como a que se segue:

        \begin{figure}[H]
          \centering
          \includesvg[width=0.12\textwidth]{assets/010b.svg}
        \end{figure}

        A tabela de expedição em questão é a seguinte:

        \begin{table}[H]
          \centering
          \begin{tabular}{l|l|l|l}
              & Destino & Custo & Encaminhar p/ \\ \hline
            1 & B       & 1     & A             \\
            2 & C       & 4     & B             \\
            3 & D       & 2     & A             \\
            4 & E       & 4     & D             \\
            5 & A       & 0     & local
          \end{tabular}
        \end{table}

        \subquestion{
  \item Suponha que, devido a atrasos na difusão de um LSP, o nó $A$ não tem conhecimento
        da ligação $D-E$, e só dela não tem conhecimento. Todos os outros nós têm conhecimento
        completo da topologia da rede. O que acontece aos pacotes enviados pelo nó $A$ com
        destino ao nó $E$? Conclua sobre o regime transitório de um protocolo estado-da-ligação.
        }

        Na tabela, a quarta linha é alterada para o tuplo \texttt{(E, 8, C)}; os pacotes
        serão enviados para $B$, que sabe que o caminho mais curto até $E$ é $BADE$,
        pelo que os pacotes serão encaminhados de volta a $A$. Como $A$ não tem conhecimento
        da ligação $D-E$, os pacotes serão enviados novamente para $B$, ficando
        assim num \textit{loop} infinito.

        \subquestion{
  \item Suponha que é estabelecida um nova ligação entre os nós $D$ e $B$ com comprimento $2$.
        No total, quantos LSPs é que vão viajar pela rede?
        }

        $B$ e $D$ apercebem-se da alteração, claro, e serão eles a fazer o primeiro
        \textit{broadcast} dos LSPs. $B$ informa $A$, $C$, e $D$, que por sua vez informam
        os seus filhos, etc. Abaixo encontra-se uma ilustração do processo:

        \begin{figure}[H]
          \centering
          \includesvg[width=0.6\textwidth]{assets/010d-tree.svg}
        \end{figure}

\end{enumerate}