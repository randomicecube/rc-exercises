\question {
  \item Considere um sistema terminal $X$ que pretende enviar um ficheiro com $100$ $K_iB$
  a um sistema terminal $Y$ através de um caminho com quatro ligações como ilustrado
  nas figuras abaixo. É usada uma tecnologia de comutação de pacotes, contendo
  cada pacote $5$ $K_iB$. Os atrasos de propagação nas ligações, bem como
  os atrasos de processamento nos nós, são desprezáveis.

  \begin{figure}[H]
    \centering
    \includesvg[width=0.7\textwidth]{assets/001.svg}
  \end{figure}
}

\begin{enumerate}[leftmargin=\labelsep]
  \subquestion{
  \item Na topologia da Figura a), cada uma das quatro ligações é cablada com ritmo de
        transmissão $10$ $Mbps$. Desenhe um diagrama espaço-tempo ilustrando a transferência
        do ficheiro e calcule o atraso na entrega do mesmo, desde a transmissão do primeiro
        bit por $X$ até à receção do último bit por $Y$.
        }

        \subquestion{
  \item Na topologia da Figura b), as ligações $(A,B)$ e $(B,C)$, e apenas
        estas, passaram a um ritmo de transmissão de $50$ $Mbps$. Repita a alínea
        anterior para este caso.
        }

        \subquestion{
  \item Na topologia da Figura c), as ligações $(A,B)$ e $(B,C)$ passaram
        a ser ligações sem-fios a $10$ $Mbps$. As antenas usadas são omnidirecionais
        (o nó $B$ não pode receber e transmitir simultaneamente) e assume-se que os
        nós $A$ e $C$ não conseguem escutar as transmissões um do outro. Desenhe um
        diagrama espaço-tempo ilustrando a transferência do ficheiro e calcule o
        atraso mínimo na entrega do mesmo, desde a transmissão do primeiro bit
        por $X$ até à receção do último bit por $Y$.
        }

\end{enumerate}