\question {
  \item Considere um sistema terminal $X$ que pretende enviar um ficheiro com $100$ $K_iB$
  a um sistema terminal $Y$ através de um caminho com quatro ligações como ilustrado
  nas figuras abaixo. É usada uma tecnologia de comutação de pacotes, contendo
  cada pacote $5$ $K_iB$. Os atrasos de propagação nas ligações, bem como
  os atrasos de processamento nos nós, são desprezáveis.

  \begin{figure}[H]
    \centering
    \includesvg[width=0.7\textwidth]{assets/001.svg}
  \end{figure}
}

\begin{enumerate}[leftmargin=\labelsep]
  \subquestion{
  \item Na topologia da Figura a), cada uma das quatro ligações é cablada com ritmo de
        transmissão $10$ $Mbps$. Desenhe um diagrama espaço-tempo ilustrando a transferência
        do ficheiro e calcule o atraso na entrega do mesmo, desde a transmissão do primeiro
        bit por $X$ até à receção do último bit por $Y$.
        }

        Em primeiro lugar, é relevante realçar que estas transmissões utilizam o
        mecanismo de \textit{Store-and-forward}: cada pacote só pode avançar quando
        todo o seu conteúdo chega ao \textit{hop} atual. Assim sendo, considerando
        três \textit{hops} intermédios entre os sistemas terminais, teremos o seguinte:

        \begin{figure}[H]
          \centering
          \includesvg[width=0.7\textwidth]{assets/001a.svg}
        \end{figure}

        Descobrir a quantidade de tempo que demora a transferir todo o ficheiro
        entre sistemas terminais é relativamente simples. É particularmente
        relevante considerar que apenas se sentem atrasos "no que ao primeiro pacote
        diz respeito": é o único com espera "inútil", onde o canal à frente está livre
        mas é obrigado a esperar. O atraso corresponde precisamente ao produto
        entre o número de \textit{hops} e o tempo de transmissão de um pacote,
        sendo que este é calculado através da expressão $L/R$, onde $L$ é a
        quantidade de bits a transmitir e $R$ é o ritmo de transmissão.
        Considerando que temos $5 K_iB = 5 \cdot 2^{10} \cdot 2^3$ bits a transmitir,
        temos:

        $$
          \begin{aligned}
            \text{Atraso} & = 3 \cdot \frac{L}{R}                                    \\
                          & = 3 \cdot \frac{5 \cdot 2^{13}}{10 \cdot 10^6} \text{ s}
          \end{aligned}
        $$

        Assim sendo, o tempo total de transmissão corresponderá à soma entre o atraso
        e o tempo de transmissão de todos os pacotes sem atraso, ou seja:

        $$
          \begin{aligned}
            \text{T} & = \text{Atraso} + N \cdot \frac{L}{R}                           \\
                     & = (3 + 20) \cdot \frac{5 \cdot 2^{13}}{10 \cdot 10^6} \text{ s} \\
                     & = 0.094208 \text{ s}
          \end{aligned}
        $$

        \subquestion{
  \item Na topologia da Figura b), as ligações $(A,B)$ e $(B,C)$, e apenas
        estas, passaram a um ritmo de transmissão de $50$ $Mbps$. Repita a alínea
        anterior para este caso.
        }

        A alteração não é particularmente significativa, afetando apenas a transmissão
        dos pacotes intra-\textit{hops}. Neste sentido, o diagrama espaço-tempo
        será o seguinte (note-se que a diferença centra-se no tempo de transmissão
        de cada pacote):

        \begin{figure}[H]
          \centering
          \includesvg[width=0.7\textwidth]{assets/001b.svg}
        \end{figure}

        Neste caso, o atraso é calculado de forma ligeiramente diferente, uma vez
        que o tempo de transmissão de cada pacote é agora variável - tanto podemos
        considerar $\frac{L}{R_1}$, com $R_1 = 10$ $Mbps$, como $\frac{L}{R_2}$,
        com $R_2 = 50$ $Mbps$. Assim sendo, e atendendo ao diagrama exposto
        acima (onde o primeiro pacote \textit{espera} duas vezes num canal com
        ritmo $R_2$ e uma num canal com ritmo $R_1$), temos:

        $$
          \begin{aligned}
            \text{Atraso} & = \frac{L}{R_1} + 2 \cdot \frac{L}{R_2}                       \\
                          & = L \cdot \left( \frac{1}{R_1} + 2 \cdot \frac{1}{R_2}\right)
          \end{aligned}
        $$

        Para além do atraso, temos ainda o tempo de transmissão de todos os pacotes,
        que é calculado da mesma forma que na alínea anterior:

        $$
          \text{T} = \text{Atraso} + N \cdot \frac{L}{R_1}
          = L \cdot \left( \frac{1}{R_1} + 2 \cdot \frac{1}{R_2} + \frac{20}{R_1} \right)
          = 0.087654 \text{ s}
        $$

        \subquestion{
  \item Na topologia da Figura c), as ligações $(A,B)$ e $(B,C)$ passaram
        a ser ligações sem-fios a $10$ $Mbps$. As antenas usadas são omnidirecionais
        (o nó $B$ não pode receber e transmitir simultaneamente) e assume-se que os
        nós $A$ e $C$ não conseguem escutar as transmissões um do outro. Desenhe um
        diagrama espaço-tempo ilustrando a transferência do ficheiro e calcule o
        atraso mínimo na entrega do mesmo, desde a transmissão do primeiro bit
        por $X$ até à receção do último bit por $Y$.
        }

        Neste caso, o diagrama espaço-tempo é o seguinte:

        \begin{figure}[H]
          \centering
          \includesvg[width=0.7\textwidth]{assets/001c.svg}
        \end{figure}

        Note-se que o atraso inicial é exatamente igual ao da primeira alínea - o primeiro pacote não vê
        alterações no seu comportamento. Contudo, os pacotes seguintes passam a ter
        comportamento "alternado", passando a poder avançar apenas quando $B$ fica
        completamente desocupado (e não quando $A-B$ fica desocupado). Assim sendo,
        o atraso adicional será, claro, maior (o dobro):

        $$
          \begin{aligned}
            \text{T} & = \text{Atraso} + 2N \cdot \frac{L}{R}                                  \\
                     & = (3 + 2 \cdot 20) \cdot \frac{5 \cdot 2^{13}}{10 \cdot 10^6} \text{ s} \\
                     & = 0.176128 \text{ s}
          \end{aligned}
        $$

\end{enumerate}