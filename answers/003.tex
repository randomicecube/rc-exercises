\question{
  \item Pretende-se estimar o tempo necessário à descarga de um documento da Web.
  O documento é constituído por um objeto HTML base que referencia três imagens de
  tamanho $5000$ byte cada. A dimensão do objeto base bem como de todos os pacotes de
  controlo é desprezável. O cliente liga-se ao servidor através de uma sessão de
  débito constante igual $1$ $Mbps$ e tempo de ida-e-volta (RTT) igual a $20$ $ms$. Qual
  o tempo necessário para descarregar o documento em cada uma das condições seguintes:
}

\begin{enumerate}[leftmargin=\labelsep]
  \subquestion{
  \item O browser usa HTTP não-persistente sem sessões TCP paralelas.
        }

        Utilizando HTTP-não persistente (e sem paralelismo de sessões), o browser
        terá forçosamente de abrir uma nova sessão TCP para cada pedido HTTP.
        Note-se que haverá um pedido inicial para o objeto base e três pedidos
        adicionais para as imagens. O diagrama espaço-tempo abaixo ilustra o
        comportamento em causa:

        \begin{figure}[H]
          \centering
          \includesvg[width=0.4\textwidth]{assets/003a.svg}
        \end{figure}

        Vão ser realizados 4 pares "SYN - SYN-ACK" e 4 pares "ACK - Resposta",
        um por pedido HTTP; assim sendo, teremos no mínimo 8 RTTs para estabelecer
        as sessões TCP. Considerando o tempo de descarga do ficheiro HTML como
        desprezável, resta adicionar o tempo de descarga das imagens (igual para
        cada uma):

        $$
          \begin{aligned}
            T_{total} & = 8 \times RTT + 3 \times \frac{5000 \times 2^3}{1 \times 10^6}               \\
                      & = 8 \times 20 \times 10^{-3} + 3 \times \frac{5000 \times 2^3}{1 \times 10^6} \\
                      & = 0.16 + 0.12                                                                 \\
                      & = 0.28 \text{ s}
          \end{aligned}
        $$

        \subquestion{
  \item O browser usa HTTP não-persistente, permitindo, no máximo, duas sessões TCP paralelas.
        }

        Neste cenário, o diagrama ficará como se segue:

        \begin{figure}[H]
          \centering
          \includesvg[width=0.45\textwidth]{assets/003b.svg}
        \end{figure}

        Assim sendo, passamos de 8 para 6 RTTs para estabelecer as sessões TCP e fazer
        os pedidos HTTP. O tempo de descarga do ficheiro HTML é desprezável.
        Duas das imagens são descarregadas em simultâneo, a metade do ritmo habitual,
        e a terceira imagem é descarregada após o término das duas primeiras (que ocorrem
        em simultâneo). Assim sendo, a quantidade de tempo pretendida é dada por:

        $$
          \begin{aligned}
            T_{total} & = 6 \times RTT + 2 \times \frac{5000 \times 2^3}{0.5 \times 10^6} + \frac{5000 \times 2^3}{1 \times 10^6}               \\
                      & = 6 \times 20 \times 10^{-3} + 2 \times \frac{5000 \times 2^3}{0.5 \times 10^6} + \frac{5000 \times 2^3}{1 \times 10^6} \\
                      & = 0.12 + 0.08 + 0.04                                                                                                    \\
                      & = 0.24 \text{ s}
          \end{aligned}
        $$

        \subquestion{
  \item O browser usa HTTP persistente com pipelining (sem sessões paralelas).
        }

        Com HTTP persistente e pipelining, o diagrama fica como se segue:

        \begin{figure}[H]
          \centering
          \includesvg[width=0.35\textwidth]{assets/003c.svg}
        \end{figure}

        Note-se que as diferenças quanto à primeira alínea centram-se no facto de que
        neste caso, o browser não precisa de abrir novas sessões TCP para cada pedido
        HTTP, e ainda em haver pipelining, permitindo a descarga das três imagens num
        único pedido. Assim sendo, o número de RTTs para estabelecer as sessões TCP passa para 3:
        o inicial para estabelecer a sessão TCP e os 4 pedidos HTTP. Com contas relativamente
        semelhantes, podemos assim chegar ao seguinte resultado:

        $$
          \begin{aligned}
            T_{total} & = 3 \times RTT + 3 \times \frac{5000 \times 2^3}{1 \times 10^6}               \\
                      & = 3 \times 20 \times 10^{-3} + 3 \times \frac{5000 \times 2^3}{1 \times 10^6} \\
                      & = 0.06 + 0.12                                                                 \\
                      & = 0.18 \text{ s}
          \end{aligned}
        $$
\end{enumerate}